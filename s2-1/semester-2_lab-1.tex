\documentclass[12pt,a4paper]{article}
\usepackage[utf8]{inputenc}
\usepackage[russian]{babel}
% \usepackage{indentfirst}
\usepackage{graphicx}
\DeclareGraphicsExtensions{.pdf,.png,.jpg}

\begin{document}
\begin{titlepage}
	\centering
	{\scshape\LARGE Лабораторна робота №3 \par}
	\vspace{1cm}
	{\scshape\Large з навчального курсу "Методи обчислень"\par}
	\vspace{1.5cm}
	{\huge\bfseries Побудова середньоквадратичного наближення функцій\par}
	\vspace{2cm}
	{\Large\itshape Чан Ха Ву\par}	
	\vfill
	прийняв роботу\par
	\textsc{Черній Дмитро Іванович}
	\vfill

% Bottom of the title page
	{\large Київ \the\year}
\end{titlepage}

% Main section
\setlength{\parindent}{3em}
\setlength{\parskip}{1em}

\section{Постановка задачі}
Дослідити метод побудови середньоквадратичного наближення для Кривої Коха, побудованої до 3-й ітерації, взявши за вузли вузли цієї кривої. Порахувати довжину побудованого елементу середньоквадратичного наближення.

\section{Математична модель та алгоритм}
	\subsection{Крива Коха}
	Крива Коха є типовим геометричним фракталом. Процес її побудови виглядає так: беремо одиничний відрізок, поділяємо на три рівні частини і замінюємо середній інтервал рівностороннім трикутником без цього сегмента. У результаті утворюється ламана, що складається з чотирьох ланок довжини $\frac{1}{3}$. На наступному кроці повторюємо операцію для кожного з чотирьох отриманих ланок і так далі. Гранична крива і є кривою Коха.
	
	\begin{figure}[!ht]
		\caption{Крива Коха, побудованої до 3-ї ітерації}
		\centering
    	\includegraphics[width=\textwidth]{koch3}
	\end{figure}	

	\subsection{Поліном найкращого середньоквадратичного наближення}
	Нехай $M_n$ – лінійна оболонка базису $\left\{ \phi_0, \phi_1, \dots \phi_n\right\}$, за яким ми шукаємо елемент найкращого середньоквадратичного наближення $\Phi_0$. Його існування для випадку гільбертового простору $H$ випливає за теоремою. Нехай ми шукаємо функцію найкращого середньоквадратичного наближення для функції $f$, $\left( \cdot, \cdot \right)$ – внутрішній добуток в даному гільбертовому просторі (у нашому випадку в якості такого простору логічно взяти простір неперервних функцій на області визначення функції $f$). Тоді за теоремою: $$\forall i \in M_n \left( f-\Phi_0, \Phi\right)=0.$$ Тоді $$\forall i \in \left\{ 0, 1, \dots n\right\} \left( f-\Phi_0, \phi_i\right)=0.$$ Нехай $$\Phi_0 = \sum_{i=0}^n {c_i \phi_i},$$ тоді $$\forall i \in \left\{ 0, 1, \dots n\right\} \left( f-\sum_{j=0}^n c_j \phi_j, \phi_i \right) = 0,$$ $$\sum_{j=0}^n {c_j \left(\phi_j, \phi_i \right)} = \left(f, \phi_i \right), i \in \left\{ 0, 1, \dots n \right\}.$$ Таким чином ми одержали систему лінійних алгебраїчних рівнянь з матрицею, що є матрицею Грамма $G = {\|\left(\phi_i, \phi_j \right)\|}_{i, j \in \left\{ 0, 1, \dots n\right\}}.$ Отже, якщо $\det\left(G\right) = 0$, то можна знайти ров'язок даної системи (в нашому випадку використовується метод Гауса), а отже і шукану функцію $\Phi_0$.
	
	\subsection{Знаходження довжини кривої графіка поліному найкращого середньоквадратичного наближення}
	Формула для знаходження довжини кривої графіка функції $\Phi_0$ на інтервалі $[0, 1]$ виглядає наступним чином: $$L = \int_a^b \sqrt{1 + \left( \frac{\partial \Phi_0}{\partial x} \right)^2}dx$$
	
	
\section{Комп’ютерне моделювання}

\begin{figure}[!ht]
	\caption{Поліном 9-го степеня найкращого середньоквадратичного наближення для Кривої Коха, побудованої до 3-й ітерації}
	\centering
   	\includegraphics[width=0.9\textwidth]{k3_p10}
	$\Phi \approx -0.00014x^9 +9.40199x^8 -75.21053x^7 +241.98971x^6 -399.01098x^5 +355.47222x^4 -165.70428x^3 +36.04233x^2 -2.48501x^1 +0.01602x^0 $ \par	
	$L  \approx 2.596208465185429$
\end{figure}	

\begin{figure}[!ht]
	\caption{Поліном 9-го степеня найкращого середньоквадратичного наближення для Кривої Коха, побудованої до 3-й ітерації}
	\centering
   	\includegraphics[width=0.9\textwidth]{k2_p8}
	$\Phi \approx 0.00000x^7 -2.80545x^6 +16.83269x^5 -37.30826x^4 +37.01513x^3 -15.82361x^2 +2.50412x^1 -0.00834x^0 + $ \par	
	$L  \approx 2.408532292853658$
\end{figure}	

\section{Висновки}
Було побудовано функцію найкращого середньоквадратичного наближення для Кривої Коха. Також було нарисовано графіки цих функції та знаходжено довжину кривої графіка. Помітно, що при збільшенні степеня полінома середньоквадратичного наближення, його графік візуально краще наближає графік наближуваної функції.
\end{document}
