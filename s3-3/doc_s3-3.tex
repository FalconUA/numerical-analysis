%%%%%%%%%%%%%%%%%%%%%%%%%%%%%%%%%%%%%%%%%
% University Assignment Title Page 
% LaTeX Template
% Version 1.0 (27/12/12)
%
% This template has been downloaded from:
% http://www.LaTeXTemplates.com
%
% Original author:
% WikiBooks (http://en.wikibooks.org/wiki/LaTeX/Title_Creation)
%
% License:
% CC BY-NC-SA 3.0 (http://creativecommons.org/licenses/by-nc-sa/3.0/)
% 
% Instructions for using this template:
% This title page is capable of being compiled as is. This is not useful for 
% including it in another document. To do this, you have two options: 
%
% 1) Copy/paste everything between \begin{document} and \end{document} 
% starting at \begin{titlepage} and paste this into another LaTeX file where you 
% want your title page.
% OR
% 2) Remove everything outside the \begin{titlepage} and \end{titlepage} and 
% move this file to the same directory as the LaTeX file you wish to add it to. 
% Then add \input{./title_page_1.tex} to your LaTeX file where you want your
% title page.
%
%%%%%%%%%%%%%%%%%%%%%%%%%%%%%%%%%%%%%%%%%

%----------------------------------------------------------------------------------------
%	PACKAGES AND OTHER DOCUMENT CONFIGURATIONS
%----------------------------------------------------------------------------------------

\documentclass[12pt]{article}
\usepackage[utf8]{inputenc}
\usepackage[russian]{babel}
\usepackage{amsmath,amsfonts,amsthm} % Math packages
\usepackage{mathtools}
\usepackage{graphicx}
\usepackage{booktabs}
\usepackage[margin=1.15in]{geometry}
\usepackage{accents}
\usepackage{url}
\usepackage{caption}
\usepackage{subcaption}
\usepackage{textcomp}
\DeclareGraphicsExtensions{.pdf,.png,.jpg}

\usepackage{lipsum} % Used for inserting dummy 'Lorem ipsum' text into the template

\usepackage{sectsty} % Allows customizing section commands
\allsectionsfont{\centering \normalfont\scshape} % Make all sections centered, the default font and small caps

\usepackage{fancyhdr} % Custom headers and footers
\pagestyle{fancyplain} % Makes all pages in the document conform to the custom headers and footers
\fancyhead{} % No page header - if you want one, create it in the same way as the footers below
\fancyfoot[L]{} % Empty left footer
\fancyfoot[C]{} % Empty center footer
\fancyfoot[R]{\thepage} % Page numbering for right footer
\renewcommand{\headrulewidth}{0pt} % Remove header underlines
\renewcommand{\footrulewidth}{0pt} % Remove footer underlines
\setlength{\headheight}{13.6pt} % Customize the height of the header

\numberwithin{equation}{section} % Number equations within sections (i.e. 1.1, 1.2, 2.1, 2.2 instead of 1, 2, 3, 4)
\numberwithin{figure}{section} % Number figures within sections (i.e. 1.1, 1.2, 2.1, 2.2 instead of 1, 2, 3, 4)
\numberwithin{table}{section} % Number tables within sections (i.e. 1.1, 1.2, 2.1, 2.2 instead of 1, 2, 3, 4)

\setlength\parindent{0pt} % Removes all indentation from paragraphs - comment this line for an assignment with lots of text

\begin{document}

\begin{titlepage}

\newcommand{\HRule}{\rule{\linewidth}{0.5mm}} % Defines a new command for the horizontal lines, change thickness here

\center % Center everything on the page
 
%----------------------------------------------------------------------------------------
%	HEADING SECTIONS
%----------------------------------------------------------------------------------------

\textsc{\Large київський національний університет імені тараса шевченка}\\[1.5cm] % Name of your university/college
\textsc{\large факультет комп'ютерних наук та кібернетики}\\[0.5cm] % Major heading such as course name
\textsc{\large кафедра прикладної математики}\\[0.5cm] % Minor heading such as course title

%----------------------------------------------------------------------------------------
%	TITLE SECTION
%----------------------------------------------------------------------------------------

\HRule \\[0.4cm]
{ \Large \bfseries Лабораторна робота №3 з курсу “Чисельні методи математичної фізики”:}\\[0.4cm] % Title of your document
{ \Large \bfseries “Розв'язання задачі теплопровідності” }
\HRule \\[1.5cm]
 
%----------------------------------------------------------------------------------------
%	AUTHOR SECTION
%----------------------------------------------------------------------------------------

\begin{minipage}{0.4\textwidth}
\begin{flushleft} \large
\emph{Студент 4-го курсу}\\
\emph{групи ОМ}\\
Чан Ха Ву % Your name
\end{flushleft}
\end{minipage}
~
\begin{minipage}{0.4\textwidth}
\begin{flushright} \large
\emph{Викладач:} \\
\emph{к.ф.-м.н., доцент} \\
Риженко \textsc{А. І.} % Supervisor's Name
\end{flushright}
\end{minipage}\\[4cm]

% If you don't want a supervisor, uncomment the two lines below and remove the section above
%\Large \emph{Author:}\\
%John \textsc{Smith}\\[3cm] % Your name

%----------------------------------------------------------------------------------------
%	DATE SECTION
%----------------------------------------------------------------------------------------

{\large Київ, 01 січня 2017}\\[3cm] % Date, change the \today to a set date if you want to be precise

%----------------------------------------------------------------------------------------
%	LOGO SECTION
%----------------------------------------------------------------------------------------

%\includegraphics{Logo}\\[1cm] % Include a department/university logo - this will require the graphicx package
 
%----------------------------------------------------------------------------------------

\vfill % Fill the rest of the page with whitespace
\end{titlepage}

\renewcommand{\refname}{Літератури та посилання}
\renewcommand{\figurename}{Мал.}

%----------------------------------------------------------------------------------------
%	ПОСТАНОВКА ЗАДАЧИ
%----------------------------------------------------------------------------------------
\section{Постановка задачі}

Цегло сферичної форми діаметром \( 2R = 20\text{мм}\), що має початкову температуру \(0 {^\circ}C\), вміщено в піч, температура якої дорівнює \(300 {^\circ}C\). Визначити, через який час температура в середині цієї цеглини дорівнюватиме \(30 {^\circ}C\). Фізичні характеристики цегла мають такі значення:

\begin{equation}
\begin{multlined} \label{text:data}
\lambda = 0,77 \,\frac{\text{Вт}}{\text{м}\cdot\text{К}}; \quad c = 0,83 \,\frac{\text{кДж}}{\text{кг}\cdot\text{К}}; \quad \rho = 1600 \,\frac{\text{кг}}{\text{м}^3}; \quad \gamma = 7 \,\frac{\text{Вт}}{\text{м}^2 \cdot \text{К}}.
\end{multlined}
\end{equation}

Тут \(c\) -- питомна щільність, \(\rho\) -- щільність, \(\lambda\) -- коефіцієнт теплопровідності, \(\gamma\) -- коефіцієнт тепловіддачі на границі.

%----------------------------------------------------------------------------------------
%	РЕШЕНИЕ
%----------------------------------------------------------------------------------------
\section{Хід розв'язку}

Процес нагрівання ідеально сферичної цеглини може бути описаний таким диференціальним рівнянням:

\begin{equation}
\begin{multlined} \label{init:problem}
c\rho\frac{\partial u}{\partial t} = \frac{\lambda}{x^2}\frac{\partial}{\partial{x}}\left(x^2\frac{\partial{u}}{\partial{x}}\right), \quad x \in (0, R), \quad t > 0
\end{multlined}
\end{equation}

Початкові та крайові умови при параметрах (\ref{text:data}) запишемо у такому вигляді:

\begin{equation}
\begin{split} \label{init:conditions}
\left.u(x, t)\right|_{t = 0} = 0 \quad \forall x \in [0, R]; \quad \left.x^2\frac{\partial{u}}{\partial{x}}\right|_{x \rightarrow 0} = 0\\ 
\left.\left(\lambda\frac{\partial{u}}{\partial{x}} + \gamma(u - u_\text{env})\right)\right|_{x=R} = 0, \quad \forall t > 0.
\end{split}
\end{equation}

де \( R = 10^{-2}\text{м}\), \(u_\text{env} = 300{^\circ}C\).

\subsection{Зведення до безрозмірного випадку}
Перетворимо задачу (\ref{init:problem}), (\ref{init:conditions}) таким чином, щоб пропали коефіцієнти теплопровідності, щільності, питомної щільності та тепловіддачі. Введемо безрозмірні змінні

\begin{equation}
\begin{split} \label{sol:newsys}
v(x, t) = \frac{(u(x, t) - u_\text{cont})}{u_\text{cont}}; \quad \hat{t} = \frac{a^2t}{R^2}, \quad \hat{x} = x/R;
\end{split}
\end{equation}

де \( a^2 = \lambda/(c\rho)\); \(u_\text{cont}\) -- деяка величина, яка відповідає, наприклад, \( u_{\text{env}}\). Переходячі в задачі (\ref{init:problem}), (\ref{init:conditions}) до безрозмірних змінних за формулами (\ref{sol:newsys}), дістаємо таку задачу для функції \( v\left(\hat{x}, \hat{t}\right)\):

\begin{equation}
\begin{split} \label{nomes:problem}
\frac{\partial{v}}{\partial{\hat{t}}} = \frac{1}{\hat{x}^2} \, \frac{\partial}{\partial \hat{x}} \left( \hat{x}^2 \, \frac{\partial v}{\partial \hat{x}} \right), \quad \hat{x} \in (0, 1), \quad t > 0
\end{split}
\end{equation}

при чому початкові та крайові умови набудуть наступний вигляд:

\begin{equation}
\begin{multlined} \label{nomes:conditions}
\left.v(\hat{x}, \hat{t})\right|_{t = 0} = v_0(\hat{x}) \quad \forall \hat{x} \in [0, 1];\\ \\
\left.\left( \frac{\partial v}{\partial \hat{x}} + \hat{\gamma}v\right)\right|_{\hat{x} = 1} = 0, \quad \left.\hat{x}^2\frac{\partial{v}}{\partial{\hat{x}}}\right|_{\hat{x} \rightarrow 0} = 0.
\end{multlined}
\end{equation}

де \( v_0(\hat{x}) = (u_0(x) - u_\text{cont})/u_\text{cont}\); \(\hat{\gamma} = \gamma R\lambda\).

\subsection{Апроксимізація різницевою схемою}

Як показано у \cite[с. 185-208]{Samarskii71}, безрозмірна задача теплопровідності у сферичних координатах (\ref{nomes:problem}), (\ref{nomes:conditions}) за допомогою інтегро-інтерполяційного метода можна апроксимувати такою системою:

\begin{equation}
\begin{multlined} \label{nomes:approx}
\Lambda(\overline{t}) y = \frac{1}{x^2}\left(x^2 y_{\,\overline{x}}\right)_x, \quad \overline{x} = x - \frac{h}{2}, \quad \overline{t} = t + \frac{h}{2}
\end{multlined}
\end{equation}


%----------------------------------------------------------------------------------------
%	ТЕОРЕТИЧЕСКИЕ ВЕДОМОСТИ
%----------------------------------------------------------------------------------------

\section{Теоретичні відомості}


%----------------------------------------------------------------------------------------
%	ПРАКТИЧНА ЧАСТИНА
%----------------------------------------------------------------------------------------


%----------------------------------------------------------------------------------------
%	РЕЗУЛЬТАТ РОБОТИ
%----------------------------------------------------------------------------------------

\section{Результат роботи програми}


\begin{thebibliography}{9}

\bibitem{TihonovSamarskii}
  А. Н. Тихонов, А. А. Самарский,
  \emph{Об однородных разностных схемах},
  Журнал вычислительной математики и математической физики, 1961.
\bibitem{Samarskii71}
  А. А. Самарский,
  \emph{Введение в теорию разностных схем},
  Главная редакция физико-математической литературы изд-ва «Наука», 
  М., 1971. 
\bibitem{SourceCode}
  Чан Х. В., \emph{Метод безпосередньої заміни диференціальних похідних частковими на мові Python 3}, 2017. 
  \url{https://github.com/FalconUA/numerical-analysis/blob/master/s3-2/appointment_s3-2.ipynb}

\end{thebibliography}
\end{document}