\documentclass[a4paper,12pt]{article}
\usepackage[utf8]{inputenc}
\usepackage[russian]{babel}
\usepackage{graphicx}
\usepackage{titlesec}
\usepackage[ruled,vlined]{algorithm2e}
\usepackage{amsthm}
\usepackage{amssymb}
\usepackage{enumitem}
\usepackage[nottoc,numbib]{tocbibind}
\usepackage{multirow}
\DeclareGraphicsExtensions{.pdf,.png,.jpg}

\begin{document}

\renewcommand{\figurename}{Мал.}

% Title page
\begin{titlepage}
	\centering
	{\scshape Київський національний університет імені Тараса Шевченка\par}
	{\scshape факультет кібернетики\par}
	{\scshape Кафедра обчислювальної математики\par}

	\vspace{2cm}
	{\scshape\LARGE Лабораторна робота №2 \par}
	\vspace{0.5cm}
	{\scshape\Large з навчального курсу "Методи обчислень"\par}
	\vspace{1cm}
	{\Large\bfseries "Моделювання руху легкого точкового тіла в системі
вихрів на площині"\par}
	\vspace{3cm}
	\begin{flushright}
	{\large Виконав:\\студент ІІІ курсу бакалаври\\\textsc{Чан Ха Ву}\par}
	\vspace{1cm}
	{\large Прийняв:\\\textsc{Черній Дмитро Іванович}\par}
	\end{flushright}
	
	\vfill
	{\large Київ \the\year}
\end{titlepage}

% Main content
\setlength{\parindent}{0em}
\setlength{\parskip}{1em}

\section{Опис методу}
У лабораторній роботі описується поведінка легкого точкового тіла, що перебуває у векторному полі швидкостей, індукованих вихрами. При цьому при обчисленні положення тіла використати формулу трапецій.

\section{Постановка задачі}
Дослідити рух легкого тіла на площині, яке перебуває у векторному полі швидкостей, що спричинене вихрами, рівномірно розташованими на осі $\textup{O}x$ і симетрично відносно центру координат. 

\section{Математична модель та алгоритми}
Точкове тіло перебуває у векторному полі швидкостей, індукованому вихрями. Вважаючи його нескінченно легким, будемо знехтувати інерцією. Вектор швидкості у кожній точці поля обчислюється за формулою:
$$ \overrightarrow{V}(x, y) = \sum_{m=1}^{M} {\overrightarrow{V}_m(x, y)},$$

де $\overrightarrow{V}_m(x, y)$ -- вектор швидкості, викликаний $m$-м вихрєю. Цей вектор обчислюється наступним чином:
$$ \overrightarrow{V}_m = (u_m(x, y), v_m(x, y)).$$

Значення $u_m(x,y)$ та $u_m(x,y)$ виражаються через координати точки та координати вихрей:
$$  u_m (x, y) = \frac{\Gamma_m}{2 \pi} \cdot \frac{Y_m - y}{R^2_m} $$
$$  v_m (x, y) = \frac{\Gamma_m}{2 \pi} \cdot \frac{x - X_m}{R^2_m} $$

Де $R_m$ -- відстань між точкою поля та $m$-м вихрею, що обчислюэться за формулою:
$$ R_m = \sqrt{(x-X_m)^2 + (y-Y_m)^2}. $$

Отже, задача моделювання траєкторії зводиться до пощуку розв'язків задачі Коші:
$$\frac{\mathrm{d} \overrightarrow{r}}{\mathrm{d} t} = \overrightarrow{V}(\overrightarrow{r})$$
$$\overrightarrow{r}(t_0) = \overrightarrow{r}_0$$

де $\overrightarrow{r}(t) = \left( x(t), y(t)\right)$ -- координати точкового заряду. Розв'язок цієї задачі можна знайти методом Ейлера або Рунге-Кутта 4-го порядку.


\subsection{Метод Ейлера}
\subsubsection{Опис методу}
На кожному кроці моделювання будемо обчислювати координати точкового тіла наступним чином:

$$ x_{n+1} = x_n + \frac{\Delta_t}{2} (u(x_n, y_n) + u(x_{n-1}, y_{n-1})) $$
$$ y_{n+1} = y_n + \frac{\Delta_t}{2} (v(x_n, y_n) + v(x_{n-1}, y_{n-1})) $$

де $\Delta_t = 0.005$, $(x_n, y_n)$ – координати тіла, $(u,v)$ – проекції вектора швидкості  на вісі $\textup{O}x$ та $\textup{O}y$ відповідно, тобто $v=(u,v)$.

\subsubsection{Оцінка похибки}
$$y_{n+1} = y_n + h \cdot f(x_n, y_n)$$
$$R_n(h) = y_{n+1} - u_{n+1} = y_n + h \cdot f(x_n, y_n) - u(x_{n+1})$$

Для оцінки локальної похибки $y_n = u_n$, розклавши $u_{n+1} = u(x_{n+1})$ в ряд Тейлора, отримаємо:

$$R_n(h) = u_n + h \cdot f(x_n, u_n) - \left( u_n + h \cdot f(x_n, y_n) + \textup{O}(h^2) \right) = \textup{O}(h^2)$$

Отже, локальна похибка методу Ейлера становить $\textup{O}(h^2)$. Порядок точності методу становить: $\frac{1}{h} (x - x_0) \textup{O}(h^2) = \textup{O}(h)$, а тому порядок апроксимації -- перший.

\subsubsection{Стійкість}
$$ y_{n+1} = y_n + h \cdot f(y_n)$$
$$ f(y_n) = \lambda \cdot y_n $$
$$ y_n = g^n $$
$$ g^{n+1} = g^n + \lambda \cdot h \cdot g^n $$
$$ g = 1 + \lambda \cdot h $$
$$ | g | \le 1 \Rightarrow |1 + \lambda \cdot h| \le 1 \Rightarrow h \le \frac{2}{|\lambda|} \textup{ при } \lambda < 0$$

Отже, метод Ейлера є умовно стійким, оскільки він не є стійким для $\forall h$. 

\subsection{Метод Рунге-Кутта 4-го порядку}
\subsubsection{Опис методу}
На кожній ітерації обчислюється наближене значення $\overrightarrow{r}$:
$$\textbf{\overrightarrow{r}}_{n+1} = \textbf{\overrightarrow{r}}_n + {h \over 6}(\textbf{k}_1 + 2\textbf{k}_2 + 2\textbf{k}_3 + \textbf{k}_4)$$

Де значення $\textbf{k}_1, \textbf{k}_2, \textbf{k}_3, \textbf{k}_4$ обчислюється наступним чином:

$$\textbf{k}_1 = \overrightarrow{V}(\overrightarrow{r}_{n-1})$$
$$\textbf{k}_2 = \overrightarrow{V}(\overrightarrow{r}_{n-1} + \frac{h}{2}\textbf{k}_1)$$
$$\textbf{k}_3 = \overrightarrow{V}(\overrightarrow{r}_{n-1} + \frac{h}{2}\textbf{k}_2)$$
$$\textbf{k}_4 = \overrightarrow{V}(\overrightarrow{r}_{n-1} + h \textbf{k}_3)$$

\subsubsection{Оцінка похибки}
Так, як для методу Рунге-Кутта 4-го порядку $( m = 4)$ виконується наступне:

$$ y_{n+1} = \sum_{i=1}^{m} {p_i k_i (h)} $$

де 
$$ k_i(h) = h \cdot f(\zeta_i, \eta_i)$$
$$ \zeta_i = x_n + \alpha_i h $$
$$ \eta_i = y_n + \beta_{i, 1}k_1(h) + \dots + \beta_{i, i-1} k_{i-1}(h), i = {1 \dots m}$$

Звідси отримуємо:

$$ R_n(h) = \sum_{k=0}^s {\frac{1}{k!} \left( h^k R_n^{(k)}(0) \right) + \frac{1}{(s+1)!} R^{(s+1)}(\theta h)} = \textup{O}(h^{s+1})$$

Отже, локальна похибка методу Рунге-Кутта 4-го порядку становить $ \textup{O}(h^5)$. Порядок точності методу становить: $ \frac{(x - x_0)}{h} \textup{O}(h^5) = \textup{O}(h^4)$, а тому порядок апроксимації -- червертий.

\subsubsection{Стійкість}
$$ y_{n+1} = \sum_{i=1}^{m} {p_i k_i (h)} $$
$$\textbf{k}_1 = \overrightarrow{V}(\overrightarrow{r}_{n-1})$$
$$\textbf{k}_2 = \overrightarrow{V}(\overrightarrow{r}_{n-1} + \frac{h}{2}\textbf{k}_1)$$
$$\textbf{k}_3 = \overrightarrow{V}(\overrightarrow{r}_{n-1} + \frac{h}{2}\textbf{k}_2)$$
$$\textbf{k}_4 = \overrightarrow{V}(\overrightarrow{r}_{n-1} + h \textbf{k}_3)$$

Маємо:
$$ f(y_n) = \lambda \cdot y_n$$
$$ y_n = g^n $$
$$ g^{n+1} = g^n \left( 1 + \lambda h + \frac{\lambda^2 h^2}{2} + \frac{\lambda^3 h^3}{6} + \frac{\lambda^4 h^4}{24}\right)$$
$$ g = 1 + \lambda h + \frac{\lambda^2 h^2}{2} + \frac{\lambda^3 h^3}{6} + \frac{\lambda^4 h^4}{24}$$
$$ |g| \le 1 \Rightarrow \lambda \cdot h \in (0, 0.754296/|\lambda|)$$

Отже, метод Рунге-Кутта є умовно стійким.

\newpage
\section{Комп’ютерне моделювання}
Нехай червоні точки позначають центри вихрів, зелена крива – слід траєкторії руху точкового тіла. У векторному полі знаходяться 5 вихрей з координатами:
$$r_1 = (-1.0, -1.0)$$
$$r_2 = (-1.0, 1.0)$$
$$r_3 = (1.0, -1.0)$$
$$r_4 = (1.0, 1.0)$$
$$r_5 = (0.0, 0.0)$$

та інтенсивностями $\Gamma_m = \Gamma = \frac{1}{5}$, де $m \in \{1, 2, \dots 5\}$. На (Мал. \ref{fig:1}) показано результат моделювання методом Ейлера.

\begin{figure}[!ht]
	\centering
   	\includegraphics[width=0.8\textwidth]{figure_3.png}
	\caption{Моделювання методом Ейлера}
	\label{fig:1}
	інтенсивність вихрей: $\Gamma_m = \Gamma = \frac{1}{5}$ \\
	початкова координата: $v_0 = (u_0, v_0) = (0.5, 0.8)$ \\
	для $t \in [0 \dots 100]$, $\Delta t = 0.005$
\end{figure}	
\newpage


\begin{figure}[!ht]
	\centering
   	\includegraphics[width=0.8\textwidth]{figure_4.png}
	\caption{Моделювання методом Ейлера}
	\label{fig:2}
	інтенсивність вихрей: $\Gamma_m = \Gamma = \frac{1}{5}$ \\
	початкова координата: $v_0 = (u_0, v_0) = (0.91, 0.91)$ \\
	для $t \in [0 \dots 100]$, $\Delta t = 0.005$
\end{figure}	

На (Мал. \ref{fig:2}) показано результат моделювання методом Ейлера для іншої початкової точки, ближчої до крайнього вихру.


\newpage
\begin{figure}[!ht]
	\centering
   	\includegraphics[width=0.8\textwidth]{figure_5.png}
	\caption{Моделювання методом Рунге-Кутта 4-го порядку}
	\label{fig:3}
	інтенсивність вихрей: $\Gamma_m = \Gamma = \frac{1}{5}$ \\
	початкова координата: $v_0 = (u_0, v_0) = (0.5, 0.7)$ \\
	для $t \in [0 \dots 100]$, $\Delta t = 0.01$
\end{figure}	

Можна побачити, що даний метод дає достатньо неточний результат та похибки накопичуються.

На (Мал. \ref{fig:3}) показано результат моделювання методом Рунге-Кутта 4-го порядку. Бачимо що результат дуже точний.


\newpage
\section{Висновки}

Було досліджено поведінку легкого точкового тіла в системі вихрів на площині для заданих випадків розподілу їх інтенсивностей. При цьому розглянуто різні кількості вихрів та різні щільності їх розміщення. Вихрі розташовані по точкм $(-1, -1), (-1, 1), (1, -1), (1, 1), (0, 0)$. Незначне відхилення моделювання можна компенсувати припущенням, що у точкового тіла є мала інерція. 
\end{document}